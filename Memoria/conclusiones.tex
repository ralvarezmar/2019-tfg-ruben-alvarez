\chapter{Conclusiones}
\label{chap:conclusiones}

Tras detallar las mejoras aportadas a \textit{WebSim}, en este capítulo se recopilan las contribuciones de este TFG, se revisa en qué medida se han alcanzado los objetivos planteados, se valoran los conocimientos adquiridos y se exponen las posibles líneas de mejora y extensión de la plataforma. 

\section{Conclusiones}
\label{sec:conclusiones}
El objetivo general de mejorar el entorno docente basado en \textit{WebSim} se ha conseguido ampliamente y con éxito en todos los frentes. Los dos más importantes son el soporte para \textit{drones} y la programación de la infraestructura necesaria para ejercicios competitivos en la plataforma. 

El primer subobjetivo consistía en añadir soporte a \textit{drone} en \textit{WebSim}. En la sección \ref{sec:drone} se explica cómo se ha llevado a cabo, aportando el modelo en 3D, \textit{drivers} y bloques para las nuevas funciones. Gracias a ello \textit{WebSim} soporta ahora la programación realista de \textit{drones}, que además de controlables en velocidad de giro y avance, como los robots en tierra, lo son también en velocidad de ascenso y descenso. Igualmente se han incluido órdenes de despegue y aterrizaje. Las físicas se han extendido para materializar movimiento en 3D incluyendo el efecto de la gravedad. También se han creado nuevos modelos de robots como fórmula 1 o \textit{mBot}, que se detallan en la subsección \ref{subsec:nuevosrobots}.\\

El segundo subobjetivo consistía en añadir teleoperadores, que se explican en la sección \ref{sec:teleoperadores}. Se ha extendido \textit{WebSim} para que acepte ficheros de configuración con los que arrancar la simulación, en los que se puede especificar los objetos del escenario y el propio robot deseado (entre los soportados). Además, se han creado archivos de configuración para poder cambiar de escenario y facilitar así su integración en servidor. Estos teleoperadores son útiles principalmente para probar y depurar el soporte a los nuevos robots en \textit{WebSim}.\\

El tercer subobjetivo era incluir más ejercicios a \textit{WebSim} y mejorar los existentes. Se han explicado con detalle en la sección \ref{sec:escenarios} y otorga a la plataforma el poder realizar nuevos ejercicios como choca-gira (subsección \ref{subsec:chocagira}), sigue-pelota (subsección \ref{subsec:pelota} ) y atraviesa-bosque (subsección \ref{subsec:atraviesabosque}).  Para todos ellos se ha creado un mundo 3D con sus objetos, obstáculos, físicas activadas, etc., y se ha programado una solución de referencia. \\

El último subobjetivo, los ejercicios competitivos, se ha descrito con detalle en la sección \ref{sec:competitive}. Para estos ejercicios se ha ampliado la funcionalidad de \textit{WebSim} refactorizando su motor de cómputo (subsección \ref{subsec:arquitectura}). En esta nueva arquitectura se separan los hilos de la simulación, cerebro del robot y editor dando la posibilidad de crear más de un robot en el mismo escenario, detener la simulación y reanudarla después de haber cambiado el código y crear un hilo de código que permita evaluar objetivamente la calidad de su comportamiento. Para su correcto funcionamiento se han refactorizado también todos editores disponibles y se han creado nuevos para los ejercicios competitivos, que se diferencian de los que había antes de este TFG en la interfaz gráfica y que dan la posibilidad de programar dos robots distintos que compiten en el mismo escenario simulado.

Para estos ejercicios se han empleado los \textit{robots} creados en la subsección \ref{subsec:nuevosrobots} y se ha incorporado un evaluador automático por cada ejercicio. El evaluador puede acceder a todos los sensores de los \textit{robots}, pero se han desarrollado para que accedan a su posición y muestren el porcentaje recorrido del circuito (ejercicios sigue-líneas y atraviesa-bosque) o la distancia con otro \textit{robot} (ejercicio gato-ratón). \\

\section{Mejoras futuras}
\label{sec:mejoras_futuras}

El desarrollo de este trabajo ha supuesto un enorme paso adelante para \textit{WebSim}, pero aún hay muchas posibles vías de desarrollo para su mejora y ampliación:

\begin{itemize}
    \item Añadir nuevos modelos de \textit{robots} como la aspiradora robótica \textit{Roomba} o un \textit{robot} con pinzas. 
    \item Añadir más ejercicios a la plataforma, en especial con un ojo puesto en los planes docentes de la asignatura ``Tecnología, Programación y Robótica'' de Educación Secundaria de la Comunidad de Madrid. Por ejemplo, aparcamiento automático o uno basado en coger objetos del entorno. Para ello habrá que construir nuevos escenarios y evaluadores automáticos. 
    \item Explorar el uso de \textit{WebWorkers} en los cerebros para optimizar el rendimiento de \textit{WebSim}. Los \textit{WebWorkers} se pueden ejecutar en diferentes \textit{cores} de la \textit{CPU} del ordenador, lo que aumentaría el rendimiento computacional de \textit{WebSim}.
    \item Explorar la gravedad que simula \textit{A-Frame} para que la simulación sea más realista cuando hay un \textit{drone} en el escenario. 
    \item Establecer un control en posición realimentado modificando la arquitectura de cómputo. Actualmente el control de posición es bloqueante (órdenes como ``\textit{avanza 1 metro}'' lleva un tiempo completarse) y actualmente se materializa con un ``\textit{sleep}'' proporcional a la distancia que se ha pedido recorrer o el ángulo a girar. Otra implementación más robusta sería materializar un bucle de control realimentado, precisamente apoyado en la medición instantánea de la distancia recorrida en cada momento, pero que no bloqueara el resto de la aplicación \textit{JavaScript}.
\end{itemize}