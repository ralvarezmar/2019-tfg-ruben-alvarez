\chapter{Objetivos y metodología}
Una vez expuestas las motivaciones y contexto del proyecto, en este capítulo se detallarán objetivos y metodología empleada. 

\label{chap:objetivos}
\section{Objetivos}
El propósito de este proyecto es la extensión y mejora de una herramienta docente para facilitar el aprendizaje de algoritmos y robótica. Para cumplir ese propósito se han fijado varios objetivos:
\begin{itemize}
    \item Añadir soporte para \textit{drone} en la plataforma, incluyendo tanto el software necesario como el modelo 3D para el entorno de \textit{A-Frame}.
    
    \item Incluir más ejercicios a \textit{WebSim}. Es necesario elaborar archivos de configuración para poder cambiar entre los distintos ejercicios y robots soportados. Esto incluye añadir más modelos de robots y nuevos escenarios a la plataforma.
    
    \item Añadir teleoperadores para que se puedan controlar los robots sin necesidad de programar. De esta manera se facilita la labor de los desarrolladores al poderse probar el entorno y los sensores del robot de manera sencilla.
    
    \item Incluir ejercicios competitivos de tal manera que dos usuarios puedan programar sobre el mismo escenario. Este objetivo también incluye crear un evaluador para puntuar el comportamiento de cada robot. 
    
\end{itemize}
\section{Metodología}
\label{sec:metodologia}



\section{GitHub}
\label{sec:github}
