\chapter{Mejoras a WebSim}
\label{chap:mejoras}
Una vez presentado el contexto, objetivos y herramientas empleadas, en este capitulo se detallan todas las mejoras realizadas del simulador \textit{WebSim} y cómo se han llevado a cabo. 

\section{Drone}
Uno de los objetivos principales del TFG era ampliar el soporte para otros robots y escenarios en \textit{WebSim}. Se ha comenzado dando soporte a drones debido a las diferencias con el \textit{piBot}, del que ya disponía soporte. 
Para ello hay que extender el \textit{software} existente de la plataforma.
\subsection{Software}
Una de las principales diferencias es el movimiento vertical, tanto como para darle velocidad al drone como para actualizar la posición. 
Se han creado las funciones para ello y se ha extendido la función \textit{updatePosition()} para poder actualizar el eje Y de la escena de \textit{WebSim}.

De esta manera, las nuevas funciones que dan soporte al drone son: 
\begin{itemize}
    \item \textit{\textbf{setL()}}: Método que permite ordenar velocidad vertical al robot. 
    \item \textit{\textbf{getV()}}: Método que devuelve la velocidad vertical del robot.
    \item \textit{\textit{\textbf{despegar}}}: Método que da velocidad vertical al robot hasta alcanzar cierta altura.
    \item \textit{\textbf{aterrizar}}: Método que da velocidad vertical negativa al robot hasta que alcance el suelo. 
\end{itemize}

Además, se ha editado la función \textit{move} para que acepte 3 parámetros. 

\begin{table}[H]
  \begin{center}
    \caption{Métodos (HAL API) de los actuadores implementados para el drone.}
    \vspace{0.5cm}
    \label{tab:tablaMotores}
    \begin{tabular}{|c|c|} 
    \hline
      \textbf{Método} & \textbf{Descripción}\\
      \hline
.setL(integer) & \begin{tabular}[c]{@{}c@{}}Mueve hacia arriba o hacia abajo el robot.\\\end{tabular} \\ \hline
.getL() & \begin{tabular}[c]{@{}c@{}}Devuelve la velocidad vertical del robot.\\\end{tabular} \\ \hline
.move(integer, integer, integer) & \begin{tabular}[c]{@{}c@{}}Mueve el robot hacia delante/atrás,\\ arriba/abajo y gira al mismo tiempo.\\ \end{tabular} \\ \hline
.despegar() & \begin{tabular}[c]{@{}c@{}}Comanda velocidad vertical al robot hasta que \\ alcanza una determinada altura.\\ \end{tabular} \\ \hline
.aterrizar() & \begin{tabular}[c]{@{}c@{}}Comanda velocidad vertical negativa al robot hasta que \\ alcanza el suelo.\\ \end{tabular} \\ \hline
    \end{tabular}
  \end{center}
\end{table}

\subsection{Bloques Scratch}
\subsection{Modelo 3D}
\subsection{Otros mundos}


\section{Teleoperadores}
\subsection{Archivos de configuración}

\section{Ejercicios competitivos}
\subsection{Escenarios}
\subsection{Evaluadores automáticos}