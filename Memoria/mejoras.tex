\chapter{Mejoras a WebSim}
\label{chap:mejoras}
Una vez presentado el contexto, objetivos y herramientas empleadas, en este capitulo se detallan todas las mejoras realizadas del simulador \textit{WebSim} y cómo se han llevado a cabo. Se va a explicar cómo se han integrado los \textit{drones} ampliando los drivers existentes para incluir nueva funcionalidad, el modelo en 3D o los bloques necesarios, los teleoperadores creados y sus ficheros de configuración. También se detallan los nuevos ejercicios creados, tanto individuales como competitivos, que incorporan evaluadores para puntuar el funcionamiento de los \textit{robots} programados.

\section{Soporte a drones en Websim}
\label{sec:drone}

Uno de los objetivos principales del TFG era ampliar el soporte para otros robots y escenarios en \textit{WebSim}. Se ha comenzado dando soporte a drones debido a las diferencias con el \textit{piBot} del que ya disponía soporte. 
Para ello hay que extender el \textit{software} existente de la plataforma.
\subsection{Drivers}
Una de las principales diferencias es el movimiento vertical o en el eje Y (en la figura \ref{fig:ejesFrame} se puede ver una representación del sistema de coordenadas de \textit{A-Frame}) tanto para darle velocidad al \textit{drone} como para actualizar la posición. 

\begin{figure}[H]
    \centering
    \includegraphics[scale=0.3]{img/ejesframe.jpg}
    \caption{Sistema de ejes en A-Frame} \label{fig:ejesFrame}
\end{figure}

Se han creado las siguiente funciones para ello:
\begin{itemize}
    \item \textit{\textbf{setL()}}: Método que permite ordenar velocidad vertical al \textit{robot}. 
    
    \begin{lstlisting}[language=javascript]
    setL(l){
        this.velocity.y = l;
    }
    \end{lstlisting}
    
    \item \textit{\textbf{getL()}}: Método que devuelve la velocidad vertical del \textit{robot}.
    
    \begin{lstlisting}[language=javascript]
getL(){
    return this.velocity.y;
}
    \end{lstlisting}
    
    \item \textit{\textit{\textbf{despegar()}}}: Método que da velocidad vertical al \textit{robot} hasta alcanzar cierta altura.
    
    
    \begin{lstlisting}[language=javascript]
  despegar(){
      var position = this.getPosition();
      while(position.y<10){
        position = this.getPosition();
        return this.velocity.y=2;
      }
      return this.velocity = {x:0, y:0, z:0, ax:0, ay:0, az:0};
    }
    \end{lstlisting}
    
    \item \textit{\textbf{aterrizar()}}: Método que da velocidad vertical negativa al \textit{robot} hasta que alcance el suelo. 
    
    \begin{lstlisting}[language=javascript]
 aterrizar(){
      var position = this.getPosition();
      while(position.y>3){
        position = this.getPosition();
        return  this.velocity = {x:0, y:-1, z:0, ax:0, ay:0, az:0};
      }
      return this.velocity.y=0;
}
    \end{lstlisting}
    
\end{itemize}

Además, se ha editado la función \textit{move} para que acepte 3 parámetros (añadiendo velocidad vertical como nuevo parámetro) y se ha extendido la función \textit{updatePosition()} para poder actualizar el eje Y en la escena de \textit{WebSim}.


\begin{lstlisting}[language=javascript, caption=Función para actualizar la posición del robot en el escenario]
    updatePosition(rotation, velocity, robotPos){
      let x = velocity.x/10 * Math.cos(rotation.y * Math.PI/180);
      let z = velocity.x/10 * Math.sin(-rotation.y * Math.PI/180);
      let y = (velocity.y/10);
      robotPos.x += x;
      robotPos.z += z;
      robotPos.y += y;
      return robotPos;
    }
\end{lstlisting}

En la tabla \ref{tab:tablaMotores2} se explican todas las funciones del \textit{HAL API} que extienden la plataforma para dar soporte al \textit{drone}.

\begin{table}[H]
  \begin{center}
    \caption{Métodos (HAL API) de los actuadores implementados para el drone.}
    \vspace{0.5cm}
    \label{tab:tablaMotores2}
    \begin{tabular}{|c|c|} 
    \hline
      \textbf{Método} & \textbf{Descripción}\\
      \hline
.setL(integer) & \begin{tabular}[c]{@{}c@{}}Mueve hacia arriba o hacia abajo el robot.\\\end{tabular} \\ \hline
.getL() & \begin{tabular}[c]{@{}c@{}}Devuelve la velocidad vertical del robot.\\\end{tabular} \\ \hline
.move(integer, integer, integer) & \begin{tabular}[c]{@{}c@{}}Mueve el robot hacia delante/atrás,\\ arriba/abajo y gira al mismo tiempo.\\ \end{tabular} \\ \hline
.despegar() & \begin{tabular}[c]{@{}c@{}}Comanda velocidad vertical al robot hasta que \\ alcanza una determinada altura.\\ \end{tabular} \\ \hline
.aterrizar() & \begin{tabular}[c]{@{}c@{}}Comanda velocidad vertical negativa al robot hasta que \\ alcanza el suelo.\\ \end{tabular} \\ \hline
    \end{tabular}
  \end{center}
\end{table}

Uno de los principales problemas encontrados es que el motor de físicas de \textit{A-Frame} no simula correctamente la posición de robot al otorgarle velocidad vertical y hace que el robot ``rebote'' sobre el escenario. Esto es debido a que el escenario tiene un atributo llamado gravedad que se aplica cada pocos milisegundos y entra en conflicto con la función \textit{updatePosition}. Se ha solucionado cambiando su valor haciendo que el escenario carezca de gravedad cuando se simula el \textit{drone}. Además, se han realizado una serie de pruebas para comprobar el comportamiento del \textit{drone}. \newline

El coste computacional aumenta al otorgarle gravedad al escenario porque hay que aumentar las iteraciones en las que el motor de físicas de \textit{A-Frame} actúa. De este manera, a más iteraciones más realista es la simulación de la gravedad, pero mayor es el coste computacional.

Los resultados obtenidos se pueden observar en la tabla \ref{tab:tablaGravedad}.

\begin{table}[H]
\caption{Pruebas de valores de gravedad e iteraciones}
\centering
\label{tab:tablaGravedad}
\begin{tabular}{|c|c|c|c|c|c|}
\hline
\multicolumn{1}{|l|}{\textbf{Gravedad}} & \multicolumn{1}{l|}{\textbf{Iteraciones}} & \multicolumn{1}{l|}{\textbf{Mín. IPS}} & \multicolumn{1}{l|}{\textbf{Máx. IPS}} & \multicolumn{1}{l|}{\textbf{Media IPS}} & \multicolumn{1}{l|}{\textbf{Coste gráfico}} \\ \hline
-4 & 30000 & 17 & 60 & 47 & 32.95\% \\ \hline
-4 & 50000 & 3 & 60 & 50 & 38.82\% \\ \hline
-2.5 & 1000000 & 3 & 60 & 51 & 41.72\% \\ \hline
-3.5 & 1000000 & 33 & 60 & 50 & 41.5\% \\ \hline
\end{tabular}
\end{table}

El mejor resultado se ha obtenido fijando la gravedad a -3.5 y las iteraciones a 1000000, de esta manera existe gravedad en el escenario y no supone un coste computacional demasiado grande. 


\subsection{Modelo 3D}

Para simular el robot en el entorno de \textit{A-Frame} es necesario realizar un modelo tridimensional. Para ello se ha buscado un modelo en una librería\footnote{\url{https://sketchfab.com/}} y se ha editado en \textit{Blender} (figura \ref{fig:droneBlender}) para que se ajuste a los requisitos del entorno. Las modificaciones que se han realizado al modelo son: 
\begin{itemize}
    \item Reducción de polinomios (modelo \textit{low-poly}) para disminuir el peso del modelo y aliviar los tiempos de carga del escenario.
    \item Rotación del modelo para que encaje con la orientación que disponía el anterior robot. Es decir, que el robot tenga su parte frontal mirando hacia el eje X positivo para que al comandarle velocidad lineal se desplace hacia delante.
    \item Modificación de la luz para adaptarla al escenario de \textit{WebSim}. 
    
    \item Elaborar una animación a las hélices para darle un aspecto más realista. Esta animación se activa vía \textit{software} cuando el drone despega del suelo\footnote{\url{https://www.youtube.com/watch?v=XjQNhNCkOJA}}.
    
\end{itemize}

 \begin{figure}[H]
    \centering
    \includegraphics[scale=0.3]{img/droneBlender.jpg}
    \caption{Drone en Blender} \label{fig:droneBlender}
\end{figure}

\textit{Blender} genera un fichero \textit{gltf} y da la posibilidad de incluir en él un binario que incluye las animaciones o exportarlo en dos distintos. En este caso, se ha hecho como un único fichero y tiene un aspecto similar a formato \textit{JSON}\footnote{\url{https://github.com/RoboticsLabURJC/2019-tfg-ruben-alvarez/blob/master/upgrades/drones/drone_animation.gltf}}. 

El drone implementado en el entorno de \textit{Websim} se puede ver en la figura \ref{fig:escenarioDrone}.

   \begin{figure}[H]
    \centering
    \includegraphics[scale=0.3]{img/websimDrone.png}
    \caption{Escenario de WebSim con drone integrado} \label{fig:escenarioDrone}
    \end{figure}

\subsection{Bloques Scratch}

%poner traduccion en javascript de los bloques 

Una vez implementado el código \textit{JavaScript} para el soporte del drone, es necesario crear los bloques con \textit{Blockly} para poder añadir sus funcionalidad en el editor de \textit{Scratch}. 

Para ello, se han creado 4 bloques con las funciones anteriormente explicadas: 
\begin{itemize}
    \item Velocidad ascenso: Comanda la velocidad ascendente del bloque que se le adjunte. 
    \begin{figure}[H]
        \centering
        \includegraphics[width=0.3\textwidth]{img/ascensionBlockly.png}
        \caption{Bloque de velocidad de ascenso} \label{fig:ascension}
    \end{figure}
    
    \item Velocidad descenso: Comanda al robot la velocidad descendente del bloque que se le adjunte. 
    \begin{figure}[H]
        \centering
        \includegraphics[width=0.3\textwidth]{img/descensoBlockly.png}
        \caption{Bloque de velocidad de descenso} \label{fig:descenso}
    \end{figure}
    \item Aterrizar: Comanda velocidad vertical al \textit{drone} hasta que alcance cierta altitud. Mantendrá esa posición hasta recibir una nueva orden.
    \begin{figure}[H]
        \centering
        \includegraphics[width=0.2\textwidth]{img/aterrizarBlockly.png}
        \caption{Bloque de aterrizaje} \label{fig:aterrizaje}
    \end{figure}
    \item Despegar: Comanda velocidad vertical negativa al \textit{drone} hasta que alcance el suelo. 
        \begin{figure}[H]
            \centering
            \includegraphics[width=0.2\textwidth]{img/despegarBlockly.png}
            \caption{Bloque de despegue} \label{fig:despegar}
        \end{figure}
\end{itemize}

 Se muestra en detalle la traducción del bloque despegar, que va incluida (junto a los otros bloques creados) en el directorio de bloques personalizados.


\begin{lstlisting}[language=javascript,label=list:traduccion]
export default function initLandBlock(){
  var landBlock = {
  "type": "land",
  "message0": "Despegar %1",
  "args0": [
    {
      "type": "field_variable",
      "name": "NAME",
      "variable": "MyRobot"
    }
  ],
  "previousStatement": null,
  "nextStatement": "String",
  "colour": "%{BKY_MATH_HUE}",
  "tooltip": "Aterriza el drone",
  "helpUrl": "Aterriza el drone"
}
  Blockly.Blocks['land'] = {
    init: function() {
      this.jsonInit(landBlock);
    }
  };
  Blockly.JavaScript['land'] = function(block) {
    var variable_name = Blockly.JavaScript.variableDB_.getName(block.getFieldValue('NAME'), Blockly.Variables.NAME_TYPE);
    var value_robotvar = Blockly.JavaScript.valueToCode(block, 'ROBOTVAR', Blockly.JavaScript.ORDER_ATOMIC);
    var code = variable_name + '.aterrizar(); \n';
    return code;
  };
  Blockly.Python['land'] = function(block) {
    var variable_name = Blockly.Python.variableDB_.getName(block.getFieldValue('NAME'), Blockly.Variables.NAME_TYPE);
    var value_robotvar = Blockly.Python.valueToCode(block, 'ROBOTVAR', Blockly.Python.ORDER_ATOMIC);

    var code = variable_name + '.aterrizar()\r\n';
    return code;
  };
}
\end{lstlisting}

Para que los bloques aparezcan en el editor de \textit{Scratch} es necesario importarlos e inicializarlos en el fichero que lo configura. En la siguiente imagen se muestra el espacio de trabajo de \textit{Scratch} con los nuevos bloques incorporados:

\begin{figure}[H]
    \centering            \includegraphics[width=0.65\textwidth]{img/kibotics_newblocks.png}
    \caption{Espacio de trabajo de \textit{Scratch} con los bloques del drone incorporados} 
    \label{fig:newblocks}
\end{figure}

\section{Teleoperadores en WebSim}
\label{sec:teleoperadores}

Se han incorporado teleoperadores en \textit{WebSim} para poder controlar los robots sin necesidad de programarlos. De esta manera es posible saber el estado y valor de sus sensores de manera sencilla ayudando a los desarrolladores a buscar fallos en drivers o incorporar nuevos escenarios. 

\subsection{Teleoperadores}


Se muestran los teleoperadores creados para los modelos del \textit{piBot} y del \textit{drone}, con la diferencia del botón de ascenso/descenso incorporado al del \textit{drone}.

\begin{figure}[H]
    \centering            \includegraphics[scale=0.25]{img/pibot_teleoperator.png}
    \caption{Teleoperador para el modelo del piBot} \label{fig:piBot_teleoperator}
\end{figure}

\begin{figure}[H]
    \centering
    \includegraphics[scale=0.25]{img/drone_teleoperator.png}
    \caption{Teleoperador para el modelo del drone} \label{fig:drone_teleoperator}
\end{figure}

Estos teleoperadores tienen una arquitectura similar a las aplicaciones creadas de \textit{Scratch} o \textit{JavaScript}. A continuación se muestra un esquema con la arquitectura de esta aplicación:

\begin{figure}[H]
    \centering            \includegraphics[scale=0.5]{img/arquitecturaTeleoperador.png}
    \caption{Arquitectura de la aplicación teleoperadores} 
    \label{fig:arq_teleop}
\end{figure}

Siendo el bloque teleoperador la interfaz creada en \textit{HTML} (listing \ref{list:teleop}). Se envía un evento cuando se pulsa uno de los botones del teleoperador o del teclado, que mediante un controlador se comunica con \textit{WebSim} tanto como para obtener las velocidades del robot como para enviarle las nuevas originadas por el evento. \textit{WebSim} se comunica con el simulador para ejecutar los nuevos valores obtenidos.

\begin{multicols}{2}
\begin{lstlisting}[language=CSS,label=list:teleopCSS]
  button {
    border-radius: 10px;
    border: none;
    color: white;
    padding: 15px 32px;
    text-align: center;
    text-decoration: none;
    display: inline-block;
    font-size: 16px;
    margin: 4px 2px;
    transition-duration: 0.1s;
    cursor: pointer;
    background-color: Transparent;
  }
.buttons{
  margin: 50px;
  position: relative;
}
#bottomArrows{
  float: center;
}
#invisible{
  visibility: hidden;
}
#upArrow{
  float: center;
}
image{
  float: center;
  border:5px solid #ff0000;
  box-sizing: border-box;
}
.buttonArrow{
  padding: 10px;
  width: 100px;
  height: 100px;
}
#land {
  display: inline-block;
  border-radius: 4px;
  background-color: #f4511e;
  border: none;
  color: #FFFFFF;
  text-align: center;
  font-size: 14px;
  padding: 10px;
  width: 100px;
  transition: all 0.5s;
  cursor: pointer;
  margin: 5px;
}
.takeland{
  float: center;
}
#takeOff {
  display: inline-block;
  border-radius: 2px;
  background-color: #32CD32;
  border: none;
  color: #FFFFFF;
  text-align: center;
  font-size: 14px;
  padding: 2px;
  width: 100px;
  transition: all 0.5s;
  cursor: pointer;
  margin: 5px;
  float: left;
}
.buttonHeight {
  background-color: #4CAF50;
  border: none;
  color: white;
  padding: 10px;
  text-align: center;
  text-decoration: none;
  display: inline-block;
  height: 100px;
  width: 100px;
  font-size: 16px;
  margin: 10px 10px;
  cursor: pointer;
  border-radius: 50%;
}
\end{lstlisting}
\end{multicols}

\begin{lstlisting}[language=html,label=list:teleop,caption=Código HTML y CSS del teleoperador del drone]
      <div id="right-container">
        <div class="buttons">
          <div id="upArrow">
              <button onclick class="buttonArrow" id="invisible"><img class="buttonArrow" src="../assets/resources/speed.svg" /></button>
              <button onclick class="buttonArrow" id="speed"><img class="buttonArrow" src="../assets/resources/speed.svg" /></button>
          </div>
          <div id="bottomArrows">
              <button onclick class="buttonArrow" id="left" ><img class="buttonArrow" src="../assets/resources/left.svg"/></button>
              <button onclick class="buttonArrow" id="brake"><img class="buttonArrow" src="../assets/resources/brake.svg"/></button>
              <button onclick class="buttonArrow" id="right"><img class="buttonArrow" src="../assets/resources/right.svg"/></button><br>
          </div>
          <div id="takeoff">
            <button onclick id="up"><img class="buttonArrow" src="../assets/resources/goUp.svg" /></button>
            <button onclick id="down"><img class="buttonArrow" src="../assets/resources/goDown.svg" /></button></button>
          </div>
        </div>
\end{lstlisting}

A continuación se detalla el código del evento que inicializa el robot, que inicializa los eventos de pulsación del teclado y botones.

\begin{lstlisting}[language=javascript,label=list:traduccion]
  document.addEventListener('robot-loaded', (evt)=>{
    localRobot = evt.detail;
    console.log(localRobot);
    document.addEventListener("keydown", keypressHandler, false);
    document.addEventListener("keyup", keyupHandler, false);
    $("#speed").click(()=>{
      speed();
    });
    $("#brake").click(()=>{
      brake();
    });
    $("#left").click(()=>{
      left();
    });
    $("#right").click(()=>{
      right();
    });
    $("#up").click(()=>{
      up();
    });
    $("#down").click(()=>{
      down();
    });
    $("#takeOff").click(()=>{
      takeOff();
    });
    $("#land").click(()=>{
      land();
    });
  });
\end{lstlisting}

Las funciones llamadas cuando se pulsa el teclado comanda una velocidad fija mientras se esté pulsando: 

\begin{lstlisting}[language=javascript,label=list:teclado]
function keypressHandler(evt){
  if (evt.key == "i"){
    localRobot.setV(0.9);
  }else if(evt.key == "l"){
    localRobot.setW(-0.2);
  }else if(evt.key == "j"){
    localRobot.setW(0.2);
  }else if(evt.key == "k"){
    localRobot.setV(-0.9);
  }else if(evt.key == "u"){
    localRobot.setL(0.9);
  }else if(evt.key == "h"){
    localRobot.setL(-0.9);
  }
}
\end{lstlisting}

Cuando se pulsa alguno de los botones de la interfaz gráfica el comportamiento es diferente que pulsando el teclado: 
\begin{itemize}
    \item Flecha hacia arriba: se obtiene la velocidad lineal del robot y se incrementa.
    \begin{lstlisting}[language=javascript,label=list:speed]
    function speed(){
      let velocity = localRobot.getV()
      localRobot.setV(velocity + 0.2);
    }
    \end{lstlisting}
    \item Flecha hacia abajo: se obtiene la velocidad lineal del robot y se reduce. 
    \begin{lstlisting}[language=javascript,label=list:brake]
   function brake(){
      let velocity = localRobot.getV()
      localRobot.setV(velocity - 0.2);
    }
    \end{lstlisting}
    \item Flechas laterales: se obtiene la velocidad angular del robot y, si es distinta de 0, se le comanda una velocidad en el sentido pulsado. 
    \begin{lstlisting}[language=javascript,label=list:right]
   function right(){
      let rotation = localRobot.getW()
      if(rotation>0){
        localRobot.setW(0);
      }else{
        localRobot.setW(-0.2);
      }
    }
    function left(){
      let rotation = localRobot.getW()
      if(rotation<0){
        localRobot.setW(0);
      }else{
        localRobot.setW(0.2);
      }
    }
    \end{lstlisting}
    \item Botón de ascenso/descenso: se obtiene la velocidad de ascensión y se incrementa/disminuye.
    \begin{lstlisting}[language=javascript,label=list:up]
        function up(){
          let velocity = localRobot.getL()
          localRobot.setL(velocity + 0.2);
        }
        function down(){
          let velocity = localRobot.getL()
          localRobot.setL(velocity - 0.2);
        }

    \end{lstlisting}
    \end{itemize}

Además de los modelos existentes del drone y \textit{piBot}, se han creado otros nuevos para extender las posibilidades de la plataforma y poder teleoperar todos ellos.
\begin{figure}[H]
    \centering            \includegraphics[scale=0.3]{img/f1_teleoperator.png}
    \caption{Teleoperador para el modelo del fórmula 1} \label{fig:f1_teleoperator}
\end{figure}

  \begin{figure}[H]
    \centering
    \includegraphics[scale=0.3]{img/mBot_teleoperator.png}
    \caption{Teleoperador para el modelo del mBot} \label{fig:mBot_teleoperator}
\end{figure}
    
También se ha creado una página principal para poder acceder a todos ellos a través de una interfaz gráfica: 

 \begin{figure}[H]
    \centering
    \includegraphics[scale=0.25]{img/teleoperators.png}
    \caption{Interfaz que permite acceder a cada uno de los teleoperadores} \label{fig:teleoperators}
\end{figure}

\begin{lstlisting}[language=html,label=list:teleop,caption=Código HTML de la interfaz para acceder a los teleoperadores]
	<body>
		<div id="wrapper">
				<header id="header" class="alt">
					<p class="logo"><strong>Teleoperators</strong></a>
				</header>
				<div id="main">
					<section id="one" class="tiles">
						<article>
							<span class="image">
								<img src="../assets/resources/drone.png" alt="drone"/>
							</span>
							<header class="major">
								<h3><a href="drone.html" class="link">Drone</a></h3>
								<p>Drone teleoperator </p>
							</header>
						</article>
						<article>
							<span class="image">
								<img src="../assets/resources/pibot.png" alt="piBot"/>
							</span>
							<header class="major">
								<h3><a href="pibot.html" class="link">PiBot</a></h3>
								<p>PiBot teleoperator</p>
							</header>
						</article>
						<article>
							<span class="image">
								<img src="../assets/resources/f1.png" alt="f1"/>
							</span>
							<header class="major">
								<h3><a href="f1.html" class="link">F1</a></h3>
								<p>F1 teleoperator</p>
							</header>
						</article>
						<article>
							<span class="image">
								<img src="../assets/resources/mBot.png" alt="mbot"/>
							</span>
							<header class="major">
								<h3><a href="mBot.html" class="link">mBot</a></h3>
								<p>mBot teleoperator</p>
							</header>
						</article>
					</section>
				</div>
		</div>
		<script src="js/jquery.scrolly.min.js"></script>
		<script src="js/jquery.scrollex.min.js"></script>
		<script src="js/browser.min.js"></script>
		<script src="js/breakpoints.min.js"></script>
		<script src="js/util.js"></script>
		<script src="js/main.js"></script>
	</body>
\end{lstlisting}

\subsection{Ficheros de configuración}

Para incorporar estos teleoperadores ha sido necesario elaborar unos ficheros de configuración y así poder cambiar el escenario y el \textit{drone} cargado por \textit{A-Frame}. 

Estos archivos de configuración se han creado en \textit{JSON} (un formato de texto para representar datos estructurados en la sintaxis de objeto de \textit{JavaScript}) y se ha programado un \textit{script} (listing \ref{list:conf}) para cargar cada fichero de configuración. Para ello se crea una variable en el \textit{index.html} del editor con la ruta en la que esté ubicado y el \textit{script} abre el fichero y recorre el \textit{JSON} para dar al escenario los valores establecidos.

En los ficheros creados se pueden configurar aspectos como el modelo del robot cargado, su posición y rotación, la posición de la cámara a bordo del robot, el fichero del cielo y suelo que debe cargar o los elementos que queramos añadir en el escenario. 
\begin{lstlisting}[language=JavaScript,caption=script que carga los ficheros de configuración,label={list:conf}]
  loadJSON(function(response) {
    var config = JSON.parse(response);
    var sceneEl = document.querySelector('a-scene');
    var robot = sceneEl.querySelector('#a-pibot');
    robot.setAttribute('gltf-model',config.robot.model);
    robot.setAttribute('scale',config.robot.scale);
    robot.setAttribute('position',config.robot.position);
    robot.setAttribute('rotation',config.robot.rotation);
    sceneEl.systems.physics.driver.world.gravity.y = config.gravity;
    sceneEl.querySelector('#ground').setAttribute('src',config.ground);
    sceneEl.querySelector('#sky').setAttribute('src',config.sky);
    sceneEl.querySelector('#ground').setAttribute('src',config.ground);
    sceneEl.querySelector('#secondaryCamera').setAttribute('position',config.secondaryCamera);
    sceneEl.querySelector('#cameraRobot').setAttribute('position',config.cameraRobot);
    if(config.objects.length>0){
      setObjects(config.objects,sceneEl);
  }
});
function setObjects(object,scene){
  for (let i in object){
    let keys = Object.keys(object[i]);
    var element = document.createElement(object[i][keys[0]]);
    for (let j = 1; j < keys.length; j++) {
      let attribute = object[i][keys[j]];
      element.setAttribute(keys[j],attribute);
    }
    scene.appendChild(element);
  }
}
\end{lstlisting}
\begin{lstlisting}[language=json]
  {
  "robot": {
    "model":"../assets/models/drone.gltf",
    "scale": "0.5 0.5 0.5",
    "position":"12 0 25",
    "rotation": "0 320 0"
  },
  "gravity": 0,
  "ground": "../assets/textures/escenarioLiso.png",
  "sky": "../assets/textures/sky.png",
  "secondaryCamera": "0 0 0",
  "cameraRobot":"0 0.03 -0.01",
  "objects":[{
      "type": "a-sphere",
      "position": "4 1 20",
      "color": "#FF0000"
      }
    ]
}
\end{lstlisting}

En este ejemplo se configura el escenario para que cargue el modelo del \textit{drone}, con el tamaño indicado en \textit{size}, la posición y rotación que aparece en \textit{position} y \textit{rotation}. En el valor \textit{gravity} se indica que el escenario no tenga gravedad, se carga la textura que posee el campo \textit{ground} como suelo del mundo y, por último, la posición de las cámaras es la ubicada en \textit{secondaryCamera} y \textit{cameraRobot}. Además, en \textit{objects} se pueden añadir todos los objetos deseados a la escena. En este ejemplo se añade una pelota de color rojo en la posición que aparece en \textit{position}.


\section{Ejercicios individuales}
\label{sec:escenarios}

Se han incorporado nuevos escenarios a \textit{WebSim} que dan la posibilidad de realizar nuevos ejercicios y  mejorar los ya disponibles. En esta sección se explicarán los escenarios con un solo robot y sus soluciones en \textit{Scratch}.

\subsection{Sigue-líneas visión}
    Se ha mejorado el escenario cambiando la textura del suelo a una creada con la trazada del circuito de Interlagos. Se ha realizado con un programa de diseño gráfico y, debido a su peso, se ha reducido posteriormente su tamaño para aliviar los tiempos de carga. 
    
    \begin{figure}[H]
    \centering
    \includegraphics[scale=0.4]{img/pibot_vision.JPG}
    \caption{Escenario para el ejercicio \textit{piBot} sigue-líneas con cámara} \label{fig:siguelineavision}
    \end{figure}
    
        \begin{figure}[H]
    \centering
    \includegraphics[scale=0.5]{img/siguelineasvisioncodigo.png}
    \caption{Solución en \textit{Scratch} para el ejercicio sigue-líneas visión} 
    \label{fig:visionSolution}
    \end{figure}
    
\subsection{Sigue-líneas infrarrojos}
    Con un recorrido similar a sigue-líneas visión, pero con fondo blanco y recorrido negro para facilitar la implementación de código al robot real y que no haya que realizar modificaciones. Para que funcionara correctamente ha sido necesario añadir el color blanco a \textit{undestandedColors} para realizar el filtro y poder pasar ``\textit{white}'' como atributo a la función \textit{getObjectColor()}.
    
    \begin{figure}[H]
    \centering
    \includegraphics[scale=0.4]{img/siguelineas_ir.JPG}
    \caption{Escenario para el ejercicio \textit{piBot} sigue-líneas infrarrojo} \label{fig:siguelineasIR}
    \end{figure}
    
           \begin{figure}[H]
    \centering
    \includegraphics[scale=0.5]{img/siguelineaIRcodigo.png}
    \caption{Solución en \textit{Scratch} para el ejercicio sigue-líneas infrarrojos} 
    \label{fig:irSolution}
    \end{figure}
    
\subsection{Choca-gira}
\label{subsec:chocagira}
Escenario creado en \textit{Blender} con un aspecto similar a su análogo en \textit{Python}. Para ello se han adaptado la mayor parte de las estructuras que dispone el escenario original para su integración en \textit{WebSim}. 

    \begin{figure}[H]
    \centering
    \includegraphics[scale=0.4]{img/bump&go.png}
    \caption{Escenario para el ejercicio choca-gira} \label{fig:chocagira}
    \end{figure}
    \begin{figure}[H]
    \centering
    \includegraphics[scale=0.5]{img/chocagiracodigo.png}
    \caption{Solución en \textit{Scratch} para el ejercicio choca-gira} 
    \label{fig:chocagiraSolution}
    \end{figure}

\subsection{Sigue-pelota}
\label{subsec:pelota}
Se ha realizado dos escenarios distintos, uno para \textit{piBot} y otro para \textit{drone}.
Ambos disponen de una pelota de color rojo, que debe ser seguida usando la cámara del \textit{robot}, a la que se le ha dado movimiento a través de primitivas de \textit{A-Frame}. En la figura \ref{fig:secuenciaDrone} se puede ver una secuencia con la animación de la pelota y en el siguiente código se muestra el archivo de configuración de este ejercicio:

\begin{lstlisting}[language=json]
{
  "robot": {
    "model":"../assets/models/drone_animation.gltf",
    "scale": "0.5 0.5 0.5",
    "position":"12 1 25",
    "rotation": "0 50 0"
  },
  "gravity": 0,
  "ground": "../assets/textures/escenarioLiso.png",
  "sky": "../assets/textures/sky.png",
  "secondaryCamera": "4 20 30",
  "cameraRobot":"0 0.03 -0.01",
  "objects":[{
      "type": "a-sphere",
      "id":"redBall",
      "position": "4 15 20",
      "color": "#FF0000",
      "radius": "1.5",
      "animation":"property: position; from: 4 15 20 ;to: 0 15 -20; dir: alternate; dur: 10000; loop: true",
      "animation__2":"property: position; from: 0 15 -20 ;to: 0 2 -20 ; delay: 10000; dir: alternate; dur: 10000; loop: true",
      "animation__3":"property: position; from: 0 2 -20 ;to: 4 2 20 ; delay: 20000; dir: alternate; dur: 10000; loop: true",
      "animation__4":"property: position; from: 4 2 20 ;to: 4 15 20; delay: 30000; dir: alternate; dur: 10000; loop: true",
      "animation__5":"property: position; from: 4 15 20 ;to: -10 15 10; delay: 40000; dir: alternate; dur: 10000; loop: true",
      "animation__6":"property: position; from: -10 15 10 ;to: 20 8 -30; delay: 50000; dir: alternate; dur: 10000; loop: true"
      }]
}
\end{lstlisting}

Haciendo especial mención al campo \textit{objects}, en el que se crea una pelota roja con la animación indicada en todos los campos \textit{animation}. Esto genera el siguiente elemento en \textit{HTML}:

\begin{lstlisting}[language=html]
<a-sphere id="redBall" position="12 1 25" color="#FF0000" radius="1.5" 
animation="property: position; from: 4 15 20 ;to: 0 15 -20; dir: alternate; dur: 10000; loop: true"
animation__2="property: position; from: 0 15 -20 ;to: 0 2 -20 ; delay: 10000; dir: alternate; dur: 10000; loop: true"
animation__3="property: position; from: 0 2 -20 ;to: 4 2 20 ; delay: 20000; dir: alternate; dur: 10000; loop: true" 
animation__4="property: position; from: 4 2 20 ;to: 4 15 20; delay: 30000; dir: alternate; dur: 10000; loop: true"
animation__5= "property: position; from: 4 15 20 ;to: -10 15 10; delay: 40000; dir: alternate; dur: 10000; loop: true"
animation__6="property: position; from: -10 15 10 ;to: 20 8 -30; delay: 50000; dir: alternate; dur: 10000; loop: true">
</a-sphere>
\end{lstlisting}

\begin{figure}[H]
\begin{subfigure}[t]{0.2\textwidth}
    \includegraphics[width=3cm, height=3cm]{img/followBallTello1.png}
\label{fig:figure2_1}
\end{subfigure}\hfill
\begin{subfigure}[t]{0.2\textwidth}
  \includegraphics[width=3cm, height=3cm]{img/followBallTello2.png}
\label{fig:figure2_2}
\end{subfigure}\hfill
\begin{subfigure}[t]{0.2\textwidth}
    \includegraphics[width=3cm, height=3cm]{img/followBallTello3.png}
\label{fig:figure2_3}
\end{subfigure}\hfill
\begin{subfigure}[t]{0.2\textwidth}
    \includegraphics[width=3cm, height=3cm]{img/followBallTello4.png}
\label{fig:figure2_4}
\end{subfigure}
\begin{subfigure}[t]{0.2\textwidth}
    \includegraphics[width=3cm, height=3cm]{img/followBallTello5.png}
\label{fig:figure2_5}
\end{subfigure}\hfill
\begin{subfigure}[t]{0.2\textwidth}
    \includegraphics[width=3cm, height=3cm]{img/followBallTello6.png}
\label{fig:figure2_6}
\end{subfigure}\hfill
\begin{subfigure}[t]{0.2\textwidth}
    \includegraphics[width=3cm, height=3cm]{img/followBallTello7.png}
\label{fig:figure2_7}
\end{subfigure}\hfill
\begin{subfigure}[t]{0.2\textwidth}
    \includegraphics[width=3cm, height=3cm]{img/followBallTello8.png}
\label{fig:figure2_8}
\end{subfigure}
\begin{subfigure}[t]{0.2\textwidth}
    \includegraphics[width=3cm, height=3cm]{img/followBallTello9.png}
\label{fig:figure2_9}
\end{subfigure}\hfill
\begin{subfigure}[t]{0.2\textwidth}
    \includegraphics[width=3cm, height=3cm]{img/followBallTello10.png}
\label{fig:figure2_10}
\end{subfigure}\hfill
\begin{subfigure}[t]{0.2\textwidth}
    \includegraphics[width=3cm, height=3cm]{img/followBallTello11.png}
\label{fig:figure2_11}
\end{subfigure}\hfill
\begin{subfigure}[t]{0.2\textwidth}
    \includegraphics[width=3cm, height=3cm]{img/followBallTello12.png}
\label{fig:figure2_12}
\end{subfigure}
\caption{Secuencia de la animación de una pelota para el ejercicio \textit{drone} sigue-pelota}
\label{fig:secuenciaDrone}
\end{figure}

    \begin{figure}[H]
    \centering
    \includegraphics[scale=0.5]{img/siguepelotacodigo.png}
    \caption{Solución en \textit{Scratch} para el ejercicio sigue pelota drone} 
    \label{fig:pelotaSolution}
    \end{figure}
    
\subsection{Atraviesa-bosque}
\label{subsec:atraviesabosque}

Creado con primitivas de \textit{A-Frame}, el escenario tiene disposición de pasillo y diversos objetos en el camino para realizar un ejercicio que haga uso del sensor de ultrasonidos para evitar obstáculos.

    \begin{figure}[H]
    \centering
    \includegraphics[scale=0.4]{img/atraviesabosque-indiv.png}
    \caption{Escenario para el ejercicio atraviesa bosque} 
    \label{fig:atraviesaBosqueind}
    \end{figure}

    \begin{figure}[H]
    \centering
    \includegraphics[scale=0.3]{img/atraviesaBosqueCodigo.png}
    \caption{Solución en \textit{Scratch} para el ejercicio atraviesa bosque} 
    \label{fig:bosqueSolution}
    \end{figure}
    

\subsection{Cuadrado con drone}

Escenario para facilitar el ejercicio ``cuadrado drone'' en el que hay que dibujar un cuadrado con el movimiento del \textit{drone}.
    
    \begin{figure}[H]
        \centering
        \includegraphics[scale=0.4]{img/cuadradoDrone.png}
        \caption{Escenario de WebSim para el ejercicio drone cuadrado} 
        \label{fig:droneCuadrado}
    \end{figure}

    \begin{figure}[H]
    \centering
    \includegraphics[scale=0.4]{img/tellocuadradocodigo.png}
    \caption{Solución en \textit{Scratch} para el ejercicio cuadrado drone} 
    \label{fig:cuadradoSolution}
    \end{figure}
    
\section{Ejercicios competitivos}
\label{sec:competitive}
Uno de los objetivos de este proyecto era añadir ejercicios competitivos. Se hace especial mención a ellos debido a que son completamente diferentes al resto de los ya creados. Este tipo de ejercicios aumenta el valor de la plataforma ya que da la posibilidad de programar dos robots en el mismo escenario, pudiendo entender la programación como un juego en el que se premia al que aporte la mejor solución.

En este tipo de ejercicios hay dos robots en una misma escena donde cada uno de ellos se puede programar con un código distinto. Para su implementación se ha creado el módulo \textit{brains} que contiene el método \textit{runBrains} que ejecuta un hilo para cada \textit{robot} existente en la escena. Además contiene los métodos \textit{stopBrain} y \textit{resumeBrain} que paran y reanudan la simulación. 
Para la integración de este tipo de ejercicios en plataforma se han incorporado dos aplicaciones más a \textit{WebSim}: ejercicios competitivos en \textit{Scratch} y ejercicios competitivos en \textit{JavaScript}. 

Se ha comenzado creando la aplicación llamada \textit{competitive-JavaScript} debido a la facilidad para probar código y hacer pruebas en el entorno. Para ello se ha cambiado la interfaz del editor, añadiendo botones para cada uno de los robots (figura \ref{fig:javascript_competitivo}) y añadir funcionalidad a cada botón para guardar el código de cada robot o mostrar el código en caso de tener uno guardado. %código del editor de JavaScript y sus botones



    \begin{figure}[H]
        \centering            \includegraphics[scale=0.30]{img/competitiveEditorJavascript.png}
        \caption{Editor de \textit{JavaScript} para ejercicios competitivos}
        \label{fig:javascript_competitivo}
    \end{figure}
    
Cada uno de los botones tiene la funcionalidad de guardar el código escrito en el editor y, si se pulsa el botón del robot que no se está editando, se guarda el código y se carga el  del otro robot, en caso de que ya haya uno guardado. Si no hay ninguno se carga un editor vacío. 

\begin{lstlisting}[language=javascript]
var editFirst = true;
var editSecond = false;
var codeFirst = null;
var codeSecond = null;
  $('#firstRobot').click(()=>{
    if(editFirst){
      codeFirst = editor.getCode();
    }
    if(editSecond){
      codeSecond = editor.getCode();
      editSecond=false;
      if(codeFirst==null){
        editor.insertCode("",editor);
      }else{
        editor.insertCode(codeFirst,editor);
      }
    }
    editFirst= true;
  });
\end{lstlisting}

Cuando se pulsa el botón de ejecutar código, se ejecuta el método \textit{runBrain} del módulo \textit{brains} obteniendo previamente el código del robot que se esté editando.
\begin{lstlisting}[language=javascript]
  $("#runbtn").click(()=>{
     if (editFirst) {
       codeFirst = editor.getCode();
     } else {
       codeSecond = editor.getCode();
     }
    if (brains.threadExists(editorRobot1)){
      if (brains.isThreadRunning(editorRobot1)){
        brains.stopBrain(editorRobot1);
        brains.stopBrain(editorRobot2);
      }else{
        brains.resumeBrain(editorRobot1,codeFirst);
        brains.resumeBrain(editorRobot2,codeSecond);
      }
    }else{
      brains.runBrain(editorRobot1,codeFirst);
      brains.runBrain(editorRobot2,codeSecond);
    }
  });
\end{lstlisting}


La aplicación \textit{competitive-Scratch} se ha realizado de  manera similar, con la diferencia de que, en este caso, es necesario guardar el código de los bloques en \textit{XML} y su traducción en \textit{JavaScript}. Para realizarlo de forma limpia se ha creado un objeto que contiene un \textit{boolean} y dos cadenas de texto (\textit{listing} \ref{lst:savecode}). En el primero indica el código de qué \textit{robot} se está editando, en una cadena de texto se guarda el código \textit{XML} y en la otra su traducción en \textit{JavaScript}. Se puede ver la interfaz de este editor en la figura \ref{fig:scratch_competitivo}.
    \begin{figure}[H]
        \centering            
        \includegraphics[scale=0.30]{img/competitivoEditorScratch.png}
        \caption{Editor de \textit{Scratch} para ejercicios competitivos}
        \label{fig:scratch_competitivo}
    \end{figure}

\begin{lstlisting}[language=javascript,label=lst:savecode, caption=código \textit{JavaScript} para guardar código de un robot]
var codeFirst = {
  js:"",
  xml:null,
  edit:true
};
var codeSecond = {
  js:"",
  xml: null,
  edit: false
};
  $('#firstRobot').click(()=>{
    if(codeFirst.edit){
      codeFirst.xml = editor.storeCode(editor.ui);
      editor.ui = editor.injectCode(editor.ui, codeFirst.xml);
    }
    if(codeSecond.edit){
      codeSecond.xml = editor.storeCode(editor.ui);
      codeSecond.edit = false;
      if(codeFirst.xml == null){
        editor.ui = editor.injectCode(editor.ui, '<xml></xml>');
      } else{
        editor.ui = editor.injectCode(editor.ui, codeFirst.xml);
      }
    }
    codeFirst.edit = true;
\end{lstlisting}

\begin{lstlisting}[language=javascript,caption=código \textit{JavaScript} para ejecutar código de los robots y guardar el que se esté editando]
$("#runbtn").click(()=>{
    if (codeFirst.edit) {
        codeFirst.xml = editor.storeCode(editor.ui);
        editor.ui = editor.injectCode(editor.ui,codeSecond.xml);
        codeSecond.js = editor.getCode();
        editor.ui = editor.injectCode(editor.ui,codeFirst.xml);
        codeFirst.js = editor.getCode();
    } else {
        codeSecond.xml = editor.storeCode(editor.ui);
        editor.ui = editor.injectCode(editor.ui,codeFirst.xml);
        codeFirst.js = editor.getCode();
        editor.ui = editor.injectCode(editor.ui,codeSecond.xml);
        codeSecond.js = editor.getCode();
    }
    if (brains.threadExists(editorRobot1)){
      if (brains.isThreadRunning(editorRobot1)){
        brains.stopBrain(editorRobot1);
        brains.stopBrain(editorRobot2);
      }else{
        brains.resumeBrain(editorRobot1,codeFirst.js);
        //agents.resumeAgent(editorRobot2,agents.code);
        brains.resumeBrain(editorRobot2,codeSecond.js);
      }
    }else{
      brains.runBrain(editorRobot1,codeFirst.js);
      //agents.runAgent(editorRobot2,agents.code);
      brains.runBrain(editorRobot2,codeSecond.js);
    }
  });
\end{lstlisting}

Para facilitar las pruebas y poder comprobar el estado de los sensores, comandar velocidades y crear nueva funcionalidad, se ha creado un escenario:
      
       \begin{figure}[H]
        \centering           
        \includegraphics[scale=0.3]{img/ejercicio_muros.png}
        \caption{Escenario para realizar pruebas con dos robots}
        \label{fig:ejercicio_muro}
    \end{figure}
    
Para puntuar el comportamiento de los robots de manera justa, se han incluido en estos ejercicios evaluadores automáticos. Van a tener diferentes comportamientos en cada ejercicio, por lo que se han desarrollado de tal forma que se pueda cargar cargar un evaluador distinto para cada uno o, incluso, no cargar ninguno. \newline

Para su implementación se ha creado la función \textit{runEvaluator}, que acepta como parámetro un \textit{array} con los identificadores de los robots y el archivo del evaluador deseado. Este fichero se recoge como variable en el \textit{index.html} del editor correspondiente (de forma similar a los archivos de configuración). 
Para llamar a \textit{runEvaluator} se comprueba que se haya pasado un fichero en el \textit{index.html} y, si eso ocurre, se realiza un \textit{require} de ese fichero y se ejecuta la función principal del evaluador, \textit{evaluator.main()}. Esta se encarga de llamar a las funciones necesarias para que el evaluador sea completo: crea la interfaz gráfica y la lógica necesaria para su correcto funcionamiento.

A continuación se explican los ejercicios y evaluadores para cada uno de ellos. 

\subsection{Atraviesa-bosque competitivo}
    
Ejercicio similar al escenario con un solo robot, pero en este caso se han creado dos pasillos con objetos para recorrerlos sin chocar con ninguno de los obstáculos. Se han añadido distintas primitivas de \textit{A-Frame} en la misma ubicación para los dos \textit{robots} para que el recorrido sea justo.

Para su evaluador en el que se crea una barra de progreso para cada robot y un cronómetro. Cuando se empiezan a mover los robots la barra de progreso se va completando y el cronómetro se inicia, para comprobar el porcentaje completado solo hay que obtener la posición del robot y compararla con el punto de llegada.

\begin{figure}[H]
\centering           
\includegraphics[scale=0.3]{img/evaluador_forest.png}
\caption{Escenario y evaluador para el ejercicio atraviesa-bosque}
\label{fig:evaluador_bosque}
\end{figure}


\subsection{Sigue-líneas competitivo}

Escenario con dos robots en el que tienen que seguir una linea de color blanco sobre fondo negro atravesando un puente en medio del circuito para que, de esta manera, ambos recorran la misma distancia.

Siendo el principal problema el puente creado porque el entorno no simulaba correctamente la malla de colisiones siendo imposible subir el puente porque el robot ``chocaba'' contra él. Se ha intentado realizar un modelo de puente en \textit{Blender}. En la figura \ref{fig:prueba_puente} se puede ver el puente creado y la malla de colisión generada por \textit{A-Frame}. 
La solución definitiva ha sido creando primitivas de \textit{A-Frame} (\textit{a-plane}) de tal forma que simule un puente y pueda cruzarlo el coche. 

    \begin{figure}[H]
        \centering            
        \includegraphics[scale=0.3]{img/prueba_puente.png}
        \caption{Prueba de puente creado en \textit{Blender}}
        \label{fig:prueba_puente}
    \end{figure}

Para este ejercicio se usa el escenario de la figura \ref{fig:siguelineas_competitivo} y su evaluador es similar al de atraviesa-bosque, con la diferencia de que es necesario guardar en todo momento la posición y distancia recorrida por cada robot para calcular el porcentaje del circuito completado.


\begin{figure}[H]
    \centering           
    \includegraphics[scale=0.25]{img/evaluator_follow_line.png}
    \caption{Ejercicio y evaluador sigue-líneas competitivo}
    \label{fig:evaluador_siguelineas}
\end{figure}

\subsection{Gato-ratón}

Con la nueva estructura de \textit{WebSim} se permite realizar el ejercicio gato-ratón con \textit{drones}. Se trata de otro tipo en el que no se compite con dos códigos programados por distintos usuarios, si no que un \textit{drone} ya está programado y el usuario solo tiene que desarrollar su solución para que el \textit{robot} no se aleje del objetivo. Para ello ha sido necesario crear un modelo al que se le ha pintado de color rojo para que sea más sencillo su filtrado. 

Para el evaluador de este ejercicio se crea un gráfico con ayuda de una librería externa de \textit{JavaScript} (\textit{JavaScript Graphics Library}\footnote{\url{http://www.jsgl.org/}}) que muestra la distancia entre \textit{drones} y el tiempo que lleva de ejecución. Se puede ver este evaluador en la figura \ref{fig:evaluador_gato_raton} así como el nuevo modelo de \textit{drone} creado para este ejercicio.
  

\begin{figure}[ht]
\centering           
\includegraphics[scale=0.3]{img/evaluador_drone.png}
\caption{Evaluador y nuevo modelo de \textit{drone} para ejercicio gato-ratón}
\label{fig:evaluador_gato_raton}
\end{figure}

Además, para este ejercicio se ha creado un \textit{script} llamado \textit{agents-methods.js} para ejecutar el código del \textit{drone} ratón sin necesidad de escribir código. Es muy similar al método \textit{runBrain} con la diferencia de que el código viene de un fichero en lugar del editor. En este módulo se guarda el código en la variable \textit{agents.code} con el siguiente código: 

\begin{lstlisting}[language=javascript]
agents.getCode = (file) => {
  var request = new XMLHttpRequest();
  request.open("GET", file);
  request.onreadystatechange = function () {
    if(request.status === 200 || request.status == 0) {
        agents.code = request.responseText;
    }
  }
  request.send();
}
agents.getCode("../assets/agents/mouse_code.js");
\end{lstlisting}

Y una vez obtenido el código se llama al método \textit{runAgent} que recoge el código e incorpora en el \textit{array} de \textit{robots} del módulo \textit{brains} la lógica programada.

\begin{lstlisting}[language=javascript]
agents.runAgent = (robotID, code) =>{
  code = 'async function myAlgorithm(){\n'+code+'\n}\nmyAlgorithm();';
  brains.threadsBrains.push({
    "id": robotID,
    "running": true,
    "iteration": brains.createTimeoutBrain(code, Websim.robots.getHalAPI(robotID), robotID),
    "codeRunning": code
  });
}
\end{lstlisting}